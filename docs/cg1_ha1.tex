
% documentation for for first practice of cg1

\documentclass{article}
\usepackage{times}
\usepackage{german}
\usepackage{fancyhdr}
\usepackage[utf8]{inputenc}

\title{Computergraphik 1. Hausaufgabe\\Dokumentation}
\author
{
	Sascha Ebert\\
	MatNr: 177182\\
	\texttt{sascha.ebert@s2006.tu-chemnitz.de}
}

\date{\today}

\begin{document}
\maketitle

\section{Anwendung}
Das Programm wird mit der folgenden Syntax aufgerufen:\\
\texttt{>> cg1\_ha1 <argument>}\\
Wobei \textless argument\textgreater\ mit einem Dateiname eines Files des Typs
cgp1 zu ersetzen ist. Der Aufruf darf derzeit nur ein Argument besitzen.

Dies erstellt nun eine Datei mit dem Schema: filename.tga
Ist jedoch eine Datei gegeben die nicht die Endung '.cg1' besitzt,
wird die Ausgabedatei unter dem Standardname 'output.tga' gespeichert.

Da das Programm unter Linux entstanden ist, ist es leider nicht möglich eine .exe-Datei
mit anzuhängen. Allerdings liegt ein makefile bei, aus welchem sie die genauen Compilier-Kommandos
entnehmen können.

\section{Erklärungen}

Ich habe mich für die C++-Implementation der Aufgabe entschieden, da es den aktuelleren
und sicheren Stand der Dinge widerspiegelt.
Im Folgenden sind die Quelltextdateien und die dazugehörigen Datenstrukturen jeweils kurz erläutert.

\subsection{cg1\_ha1.h}
In dieser Datei sind die Klassen-\emph{Deklarationen} für die Klassen \texttt{cgp1} und \texttt{puzzleBlock} plaziert.
Des weiteren finden sich hier Typendefinitionen, um mit den von ihnen vorgegebenen Spezifikationen besser umgehen
zu können (UBYTE, USHORT, UINT)

\subsubsection{Class: cgp1}
Die Klasse cgp1 repräsentiert das gleichnamige Format. Sie beinhaltet eine Methode zum lesen des Eingabe-Files,
welche die Membervariablen der Klasse mit den korrespondierenden Informationen füttert. Zum schreiben wird die Methode
\texttt{writeToTga()} verwendet. Als drittes wichtiges Element enthält die Klasse ein Feld von Zeigern auf
puzzleBlock-Instanzen welche die einzelnen Puzzelblöcke verwalten.

\subsubsection{Class: puzzleBlock}
Wie oben genannt, wird diese Klasse benutzt um die relevanten Informationen für einzelne Puzzelblöcke zu speichern.
Das wichtigste Element hier ist das 2-Dimensionale Feld \texttt{UBYTE **m\_pPixelField}\ um die Pixeldaten abzulegen.
Dabei wird das Zeigerkonzept verwendet um die Daten dynamisch im Heap zu organisieren.

\subsection{cg1\_ha1.cpp}
In dieser Quellcodedatei werden die Methoden, Initiatoren und Destruktoren für die 2 Haupt-Klassen
\footnote[1]{Meines Erachtens ist es nach ISO-Standard vorgesehen für jede Klasse ein Header-File und ein
.cpp-File zu benutzen.} \emph{definiert}.
Allerdings war es nötig einen Destuktor manuell aufzurufen um Valgrind\footnote[2]{Programm zum finden von Memory-Leaks}
zufrieden zu stellen.

Des weiteren sind jeweils print-Funktionen definiert, welche ich zum Kontrollieren des Lesevorgangs verwendet habe. Diese
werden allerdings nicht mehr aufgerufen.

\subsection{cg1\_ha1\textunderscore main.cpp}
In dieser Quelldatei wird die Main-Funktion des Programmes definiert. Genauer werden hier die Programmargumente ausgewertet
und unter Nicht-Verwendung der string-libary (weder c noch c++) recht umständlich der Dateiname umgeschrieben um die Endung
\texttt{.cg1} in \texttt{.tga} zu wandeln.

Ich bitte vielmals um Entschuldigung ihr Benennungsschemata nicht eingehalten zu haben, aber - ohne verbesserlich sein zu wollen -
eine main-Funktion gehöhrt nun einmal in eine separate Datei. Standards sind ja nu mal da um eingehalten zu werden.

\end{document}


