
% documentation for for first practice of cg1

\documentclass{article}
\usepackage{times}

\title{Computergraphik 1. Hausaufgabe\\Dokumentation}
\author
{
	Sascha Ebert\\
	MatNr: 177182\\
	\texttt{sascha.ebert@s2006.tu-chemnitz.de}
}

\date{\today}

\begin{document}
\maketitle

\section{Anwendung}
Das Programm wird mit der folgenden Syntax aufgerufen:\\
\texttt{cg1\_ha1 <argument>}

Der Aufruf darf derzeit nur ein Argument besitzen.
Dies erstellt nun eine Datei mit dem Schema: filename.tga
Ist jedoch eine Datei gegeben die nicht die Endung ".cg1" besitzt,
wird die Ausgabedatei unter dem Standardname "output.tga" gespeichert.


\section{Erkl\"arungen}

Das Programm ist unter Linux entstanden, dass es nicht m\"oglich ist eine .exe-Datei
mit anzuh\"angen.
Ich habe mich weiterhin f\"ur die C++-Implementation der Aufgabe entschieden, da es den aktuelleren
und sicheren Stand der Dinge widerspiegelt.
Im Folgenden sind die Quelltextdateien jeweils kurz erläutert.

\subsection{cg1\_ha1.h}
lala
\subsection{cg1\_ha1.cpp}
lala
\subsection{cg1\_ha1\textunderscore main.cpp}
lala


\end{document}


